\include{settings}

\begin{document}    % начало документа

    % Титульная страница
    \include{titlepage}

    % Содержание
    \include{ToC}


    \section{Введение}
    Windows API (Application Programming Interfaces) — набор базовых функций интерфейсов программирования приложений операционных систем Microsoft Windows. Предоставляет прямой способ взаимодействия приложений пользователя с операционной системой Windows. Для создания программ, использующих Windows API, корпорация Microsoft выпускает комплект разработчика программного обеспечения, который называется Platform SDK (Software Development Kit) и содержит файлы заголовков, библиотеки, примеры, документацию и инструменты, необходимые для разработки приложений для Microsoft Windows. В данной работе выполнен обзор структуры Win API.

    \section{Структура Win API}
    Функции, предоставляемые Windows API, могут быть поделены на восемь категорий.
    \begin{itemize}
        \item Системные службы (Base Services)
        \item Взаимодействие с реестром (Advanced Services)
        \item Взаимодействие с графическими устройствами ввода-вывода (Graphics Device Interface)
        \item Взаимодействие с пользовательским интерфейсом (User Interface)
        \item Взаимодействие с диалоговыми окнами (Common Dialog Box Library)
        \item Взаимодействие с элементами управления: табы, прогресс-бары и т. д. (Common Control Library)
        \item Windows Shell
        \item Сетевые службы: NetBIOS, Winsock и т. д. (Network Services)
    \end{itemize}

    А также:
    \begin{itemize}
        \item Взаимодействие с браузером и декодирование URL (Web)
        \item Мультимедиа: воспроизведение звуков, графика, взаимодействие с джойстиками и геймпадами
        \item Межпрограммное взаимодействие
        \item Библиотеки, упрощающее взаимодействие с Win API (Wrapper libraries)
    \end{itemize}
    Каждый из пунктов будет рассмотрен подробнее.

    \subsection{Системные службы}

    Системные службы предоставляют доступ к основным ресурсам системы Windows, например, к файловой системе, системным процессам, потокам и обработке ошибок.
    Эти функции находятся в файлах kernel.exe, krnl286.exe или krnl386.exe в 16-битной Windows, kernel32.dll и KernelBase.dll в 32- и 64-битной Windows.
    Данные файлы расположены в директории «System32», которая находится в каталоге «Windows» в 64-разрядной версии Windows, и в каталоге «Windows» в 16-разрядных версиях Windows.

    \subsection{Взаимодействие с реестром}

    Данные службы предоставляют доступ к функциям вне ядра.
    К ним относится реестр Windows, выключение / перезагрузка системы (или прерывание), запуск / остановка / создание службы Windows, управление учетными записями пользователей.
    Эти функции находятся в advapi32.dll и advapires32.dll в 32-битной Windows.

    Реестр Windows — это иерархическая база данных параметров и настроек в большинстве операционных систем Microsoft Windows.
    Содержит информацию и настройки для программного и аппаратного обеспечения и учетных записей пользователей.

    Реестр был введён для упорядочения информации, хранившейся до этого во множестве INI-файлов, обеспечения единого механизма чтения-записи настроек и избавления от проблем коротких имён, отсутствия разграничения прав доступа и медленного доступа к ini-файлам, хранящимся на файловой системе FAT16, имевшей серьёзные проблемы быстродействия при поиске файлов в каталогах с большим их количеством.

    Со временем (окончательно — с появлением файловой системы NTFS) проблемы, решавшиеся реестром, исчезли, но реестр остался из-за обратной совместимости и присутствует во всех версиях Windows, включая последнюю.
    Поскольку сейчас не существует реальных предпосылок для использования подобного механизма, Microsoft Windows — единственная (не считая ReactOS и eComStation) операционная система из используемых сегодня, в которой используется механизм реестра операционной системы.

    \subsection{Graphics Device Interface}
    Раздел API, посвященный графическому интерфейсу, предоставлет функции для вывода изображений на мониторы, принтеры и другие устройства вывода.
    Он находится в gdi.exe в 16-битной Windows и gdi32.dll в 32-битной Windows в пользовательском режиме.
    Поддержка GDI в режиме ядра обеспечивается win32k.sys, который напрямую взаимодействует с графическим драйвером.

    \subsection{User Interface}
    Предоставляет функции для создания и управления экранными окнами и большинством основных элементов управления, таких как кнопки и полосы прокрутки, прием ввода с клавиатуры и мыши, а также другие функции, связанные с частью графического интерфейса пользователя (GUI) Windows.
    Данная функциональная единица находится в user.exe в 16-битной Windows и user32.dll в 32-битной Windows.
    Начиная с версий Windows XP, основные элементы управления находятся в файле comctl32.dll вместе с общими элементами управления (Common Control Library).

    \textbf{Windows USER} — это компонент операционной системы Microsoft Windows, который обеспечивает базовые функциональные возможности для создания простых пользовательских интерфейсов.
    Компонент существует во всех версиях Windows и включает в себя функциональность для управления окнами, передачи сообщений, обработки ввода и стандартных элементов управления.

    Windows USER обеспечивает большую часть основного пользовательского опыта для Microsoft Windows:

    \begin{itemize}
        \item Вызов окна для рисования (используя GDI как независимый от устройства API для рисования примитивов)
        \item Затенение перекрывающихся окон позади других
        \item Размер окна и расположение
        \item Предоставление всех стандартных элементов управления окнами (таких как закрывающие окна или строки заголовка)
        \item Предоставление стандартной строки меню Windows
        \item Предоставление стандартных элементов управления (таких как кнопка, список или поле редактирования)
        \item Обеспечение управления диалоговыми окнами (сочетания клавиш, обработка клавиш табуляции)
        \item Обработка всего пользовательского ввода с помощью мыши и клавиатуры
        \item Фоновое изображение рабочего стола
        \item Рисование всех стандартных визуальных элементов
        \item Межпроцессное взаимодействие с использованием динамического обмена данными
        \item Отображение и управление курсором мыши
        \item Передача данных (буфер обмена)
    \end{itemize}

    \subsection{Common Dialog Box Library}
    Предоставляет приложениям стандартные диалоговые окна для открытия и сохранения файлов, выбора цвета и шрифта и т. д.
    Библиотека находится в файле с именем commdlg.dll в 16-битной Windows и comdlg32.dll в 32-битной Windows.

    \subsection{Common Control Library}
    Предоставляет приложениям доступ к некоторым расширенным элементам управления, предоставляемым операционной системой.
    К ним относятся статус-бары, прогресс-бары, панели инструментов и вкладки.
    Библиотека находится в файле динамически подключаемой библиотеки (DLL) commctrl.dll в 16-битной Windows и comctl32.dll в 32-битной Windows.

    \subsection{Windows Shell}
    Компонент Windows API позволяет приложениям получать доступ к функциям, предоставляемым оболочкой операционной системы, а также изменять и дополнять их.
    Компонент находится в shell.dll в 16-битной Windows и shell32.dll в 32-битной Windows.

    \subsection{Сетевые службы}
    Предоставляют доступ к различным сетевым возможностям операционной системы.
    Его подкомпоненты включают в себя NetBIOS, Winsock, NetDDE, удаленный вызов процедур (RPC) и многие другие.
    Этот компонент находится в netapi32.dll в 32-битной Windows.

    \subsection{Web}
    Встроенный веб-браузер (ранее Internet Explorer) также предоставляет множество API-интерфейсов, которые часто используются приложениями и поэтому могут рассматриваться как часть API-интерфейса Windows.
    IE включен в операционную систему начиная с Windows 95 OSR2 и предоставляет веб-сервисы приложениям начиная с Windows 98.\linebreak

    В частности, он используется для предоставления:

    \begin{itemize}
        \item Встраиваемого элемента управления веб-браузера, содержащегося в shdocvw.dll и mshtml.dll.
        \item Службы URL-имен, хранящейся в urlmon.dll, которая предоставляет COM-объекты приложениям для разрешения URL-адресов. Приложения могут также предоставлять свои собственные обработчики URL.
        \item Клиентской библиотеки HTTP, которая также учитывает общесистемные настройки прокси (wininet.dll).
        \item Библиотеки для поддержки многоязычной поддержки текста (mlang.dll).
        \item DirectX Transforms.
        \item Поддержки XML (компоненты MSXML, хранящиеся в msxml * .dll).
        \item Доступа к адресным книгам Windows.
    \end{itemize}

    \subsection{Мультимедиа}
    Стандартный Windows Multimedia API находится в winmm.dll и содержит функции для воспроизведения звуковых файлов, для отправки и получения MIDI-сообщений (Musical Instrument Digital Interface), для доступа к джойстикам и т. д.
    Кроме того, начиная с Windows 95 OSR2, Microsoft начала предоставлять как часть каждой версии Windows API-интерфейсы DirectX — набор графических и игровых сервисов, включающий в себя:

    \begin{itemize}
        \item Direct2D для аппаратного ускорения двухмерной векторной графики.
        \item Direct3D для аппаратного ускорения 3D-графики.
        \item DirectSound для низкоуровневого аппаратного ускорения доступа к звуковой карте.
        \item DirectInput для связи с устройствами ввода, такими как джойстики и геймпады.
        \item DirectPlay как многопользовательская игровая инфраструктура. Начиная с DirectX 9 этот компонент устарел, и Microsoft больше не рекомендует использовать его для разработки игр.
        \item DirectDraw для 2D-графики в более ранних версиях DirectX, теперь устарел и заменен Direct2D.
        \item WinG для 2D-графики в 16-битных играх, написанных для версий Windows 3.x. Устарел с выпуском Windows 95.
    \end{itemize}
    Microsoft также предоставляет несколько API для кодирования и воспроизведения медиа:

    \begin{itemize}
        \item DirectShow, который создает и запускает универсальные мультимедийные конвейеры. Он сопоставим с платформой GStreamer и часто используется для рендеринга игровых видеороликов и создания медиаплееров (на его основе основан проигрыватель Windows Media). DirectShow больше не рекомендуется для разработки игр.
        \item Media Foundation, новый API для цифровых медиа, призванный заменить DirectShow.
    \end{itemize}
    \subsection{Межпрограммное взаимодействие}
    Windows API разработан главным образом для взаимодействия между приложением и операционной системой.
    Однако для связи приложений между собой Microsoft разработала ряд технологий наряду с основным Windows API. К таким разработкам относятся:
    \begin{itemize}
        \item Dynamic Data Exchange (DDE)
        \item Object Linking and Embedding (OLE)
        \item Component Object Model (COM)
        \item ActiveX
        \item .NET Framework
    \end{itemize}

    У перечисленных технологий много общего.
    Некоторые из них, например DDE, уже устарели.
    Примером межпрограммного взаимодействия может быть передача изображения для изменения из текстового редактора в фото-редактор и возвращение назад.
    \subsection{Wrapper libraries}
    Корпорация Microsoft разработала различные библиотеки, взявшие на себя некоторые функции Windows API более низкого уровня и позволившие приложениям взаимодействовать с API более абстрактным способом.

    Microsoft Foundation Class Library (MFC) инкапсулировала функциональность Windows API в классы C ++ и, таким образом, предоставила возможность объектно-ориентированного взаимодействия с API.
    Большинство фреймворков для разработки под Windows включают в себя Windows API, так становится возможной разработка приложений под Windows с помощью других языков, например, .NET Framework или Java.
    \section{Выводы}
    Была рассмотрена структура Windows API, включающая в себя восемь разделов: системные службы, взаимодействие с реестром, взаимодействие с графическими устройствами ввода-вывода, взаимодействие с пользовательским интерфейсом, взаимодействие с диалоговыми окнами, взаимодействие с элементами управления, сетевые службы и Windows Shell.
    Также в Win API входит взаимодействие с веб-службами, мультимедиа (поддержка аппартаного ускорения графики и воспроизведения звуковых файлов) и межпрограммное взаимодействие. Wrapper Libraries (библиотеки, инкапсулирующие в себя Win API) позволяют разрабатывать приложения под Windows на других языках программирования.
    \begin{thebibliography}{9}
        \bibitem{Wikipedia} Материал из Википедии — свободной энциклопедии [Электронный ресурс], Windows API, Режим доступа: \url{https://ru.wikipedia.org/wiki/Windows_API}, свободный.
        \bibitem{MSDN} Каталог API (Microsoft) и справочных материалов [Электронный ресурс], Режим доступа: \url{https://msdn.microsoft.com/library}, свободный.
    \end{thebibliography}

\end{document}
