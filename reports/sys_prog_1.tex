\include{settings}

\begin{document}    % начало документа

    % Титульная страница
    \include{titlepage}

    % Содержание
    \include{ToC}


    \section{Цель работы}
    Освоить механизмы структурной обработки исключений в операционной системе Windows и реализовать различные их варианты на языке С++ с помощью WinAPI.

    \section{Программа работы}
    Выполнить задания для исключений двух типов:
    \begin{itemize}
        \item Ошибка деления на ноль (EXCEPTION\_INT\_DIVIDE\_BY\_ZERO)
        \item Переполнение разряда Integer (EXCEPTION\_INT\_OVERFLOW)
    \end{itemize}

    Задания для каждого пункта:
    \begin{enumerate}
        \item Сгенерировать и обработать исключения с помощью функций WinAPI
        \item Получить код исключения с помощью функции \textbf{GetExceptionCode}.
        \begin{itemize}
            \item Использовать эту функции в выражении фильтра;
            \item Использовать ту функци в обработчике;
        \end{itemize}
        \item Создать собственную функцию-фильтр;
        \item Получить информацию об исключении с помощью функции \textbf{GetExceptionInformation}; сгенерировать исключение с помощью функции \textbf{RaiseException};
        \item Использовать функции \textbf{UnhandledExceptionFilter} и \textbf{SetUnhandledExceptionFilter} для необработанных исключений;
        \item Обработать вложенные исключения;
        \item Выйти из блока \_\_try с помощью оператора \textbf{goto};
        \item Выйти из блока \_\_try c помощью оператора \textbf{leave};
        \item Преобразовать структурное исключение в исключение языка С, используя функцию \textbf{translator};
        \item Использовать финальный обработчик \textbf{finally};
    \end{enumerate}

    Для каждого пункта представить отдельную программу.
    Специфический код, связанный с особенностями генерации заданного исключения, структурировать в отдельный элемент (функцию, макрос или иное).

    \section{Теоретическая информация}
    Исключение — это событие при выполнении программы, которое приводит к её ненормальному или неправильному поведению.
    Существует два вида исключений: аппаратные, которые генерируются процессором, и программные, генерируемые операционной системой и прикладными программами.

    В операционной системе Windows механизм обработки исключений выполняется:
    \begin{itemize}
        \item Средствами языка С++
        \item Средствами Windows — Structured Exception Handling (SEH)
    \end{itemize}
    Между данными механизмами существует ряд отличий:
    \begin{itemize}
        \item SEH пригоден как для программных, так и для аппаратных исключений в отличие от средств С++
        \item В SEH исключение — это ошибка при выполнении программы, в С++ — это объект произвольного типа. Обработчик catch может рассматриваться как функция с одним параметром, которая выполняется только при совпадении типа ее параметра с типом выброшенного исключения.
        \item По-разному раскручивается стек
    \end{itemize}
    Суть механизма SEH заключается в следующем:
    \begin{itemize}
        \item В программе выделяется блок кода — фрейм, в котором может произойти исключение (охраняемый код);
        \item За фреймом размещается блок обработчика;
        \item Далее размещается первая инструкция, исполняемая после выполнения обработчика.
    \end{itemize}
    Механизм SEH поддерживается Microsoft только на уровне компилятора с помощью реализации нестандартных синтаксических конструкций \_\_try, \_\_except и \_\_finally.
    Ключевое слово \_\_try используется для выделения участка кода, в котором генерация исключения будет обработана одним или несколькими блоками \_\_except.
    Код, находящийся в блоке \_\_finally, выполнится всегда и независимо от других блоков \_\_try и \_\_except

    \begin{lstlisting}[caption=Пример использования с языке С/С++]
        __try {
        // защищенный код,
        // который помещается в SEH-фрейм
        }
        __except (фильтр исключений) {
        // обработчик исключений
        }
        __finally {
        // выполняющийся в любом случае код
        }
    \end{lstlisting}

    В качестве фильтра исключений могут выступать обычные функции, возвращающие три константных выражения:
    \begin{enumerate}
        \item EXCEPTION\_EXECUTE\_HANDLER — указывает на возможность данного обработчика обработать исключение. При получении такого значения операционная система прекращает поиск релевантных обработчиков исключения и, выполнив раскрутку стека, передаёт управление первому обработчику, вернувшему значение;
        \item EXCEPTION\_CONTINUE\_EXECUTION — указывает на исправление ошибки. Система снова передаст управление на инструкцию, которая вызвала исключение, поскольку предполагается, что в этот раз она не вызовет исключение;
        \item EXCEPTION\_CONTINUE\_SEARCH — указывает, что подходящий обработчик может быть найден выше по стеку. В то же время возвращение этого значения может быть свидетельством того, что ошибка не обработана.
    \end{enumerate}

    \section{WinDbg}
    Отладка — это процесс поиска и устранения дефектов или проблем в компьютерной программе, которые мешают правильной работе программного обеспечения или системы.
    В данной работе для изучения особенностей исключений использовался отладчик WinDbg.
    WinDbg — это многоцелевой отладчик для Microsoft Windows ОС, распространяемый Microsoft.
    Его можно использовать для отладки приложений пользовательского режима, драйверов устройств и самой операционной системы в режиме ядра.
    WinDbg может использоваться для отладки дампов памяти в режиме ядра, созданных после того, что обычно называют «синим экраном смерти», возникающим при выполнении проверки на наличие ошибок.
    Его также можно использовать для отладки аварийных дампов в пользовательском режиме.
    WinDbg может автоматически загружать файлы отладочных символов (например, файлы PDB) с сервера, сопоставляя различные критерии (например, временную метку, CRC, одиночную или многопроцессорную версию) через SymSrv (SymSrv.dll).
    Последние версии WinDbg были и распространяются как часть бесплатного комплекта средств отладки для Windows, который имеет общий отладочный интерфейс между WinDbg и внешними интерфейсами отладчика командной строки, такими как KD, CDB и NTSD.
    Большинство команд можно использовать как есть со всеми включенными интерфейсами отладчика.

    WinDbg имеет следующие команды и операции, отвечающие за контроль исполнения процесса отладки и анализа кода на наличие ошибок:
    \begin{enumerate}
        \item g – продолжить выполнение;
        \item p – шаг через функцию;
        \item t – шаг внутрь функции;
        \item pa addr – шаг в адрес;
        \item pc – шаг в следующий вызов;
        \item pt – шаг к следующему возврату;
        \item pct – шаг к следующему вызову или возврату;
        \item bp – установка точки останова, например bp nt!NtCreateFile;
        \item bl – список точек останова;
        \item bd – <число> убрать точку останову под номером;
        \item bc – <число> очистить точку останова под номером;
        \item ba – точка останова на доступ;
        \item be – точка останова на исполнение;
        \item be – точка останова на исполнение;
        \item bw – точка останова на запись;
        \item sxe ld:kernel32 – точка останова на загрузке DLL модуля;
    \end{enumerate}
    \section{Ход выполнения работы}

    \subsection{Обработка деления на ноль}
\subsubsection{Генерация и обработка исключений с помощью функций WinAPI}

\begin{lstlisting}[caption=Генерация и обработка исключения]
#include "stdafx.h"
#include <iostream>
#include <windows.h>
#include <exception>

int main()
{
    __try {
        int x, y = 0;
        x = 5 / y;
    } __except (EXCEPTION_EXECUTE_FAULT) {
        std::cout << "Division by zero";
    }

    return 0;
}
\end{lstlisting}

\begin{figure}[H]
    \begin{center}
        \includegraphics[width=1\columnwidth]{fig/Lab1/ZeroDiv/1_1.png}
        \caption{Результат работы программы}
        \label{pic:1_1}
    \end{center}
\end{figure}

\begin{figure}[H]
    \begin{center}
        \includegraphics[width=1\columnwidth]{fig/Lab1/ZeroDiv/1_2.png}
        \caption{Информация об обработке исключения в Windbg}
        \label{pic:1_2}
    \end{center}
\end{figure}

Из рисунка 5.2 видно, что исключение деления на ноль успешно обработано (first chance).

\subsubsection{Получение кода исключения}
Программа реализует отображение кода исключения двумя способами: в выражении фильтра и в блоке \_\_except обработчика прерывания.
За получение кода исключения отвечает функция \textbf{GetExceptionCode()}

В листинге №3 показана работа с данной функцией, которая используется в выражении-фильтре блока \textbf{\_\_except} и непосредственно в самом блоке \textbf{\_\_except}.
\begin{lstlisting}[caption=Получение кода исключения с помощью функции GetExceptionCode]
#include "stdafx.h"
#include <iostream>
#include <windows.h>
#include <exception>
#include <winerror.h>

int main()
{
    DWORD code = -1;
    __try {
        int x, y = 0;
        x = 5 / y;
    }
    __except (code = GetExceptionCode(), ((code == EXCEPTION_INT_DIVIDE_BY_ZERO) ? EXCEPTION_EXECUTE_HANDLER : EXCEPTION_CONTINUE_SEARCH)) {
        std::cout << "Division by zero inside except, exception code " << code;
    }

    code = -1;

    __try {
        int x, y = 0;
        x = 5 / y;
    }
    __except (code = GetExceptionCode(), EXCEPTION_EXECUTE_HANDLER) {
        switch (code) {
            case EXCEPTION_INT_DIVIDE_BY_ZERO:
                std::cout << "\nDivision by zero inside filter, exception code " << code;
                break;
            default:
                EXCEPTION_CONTINUE_SEARCH;
        }
    }

    std::getchar();

    return 0;
}
\end{lstlisting}

\begin{figure}[H]
    \begin{center}
        \includegraphics[width=1\columnwidth]{fig/Lab1/ZeroDiv/2_1.png}
        \caption{Результат работы программы}
        \label{pic:2_1}
    \end{center}
\end{figure}

\begin{figure}[H]
    \begin{center}
        \includegraphics[width=1\columnwidth]{fig/Lab1/ZeroDiv/2_2_1.png}
        \caption{Информация об обработке исключения в Windbg. GetExceptionCode в выражении-фильтре}
        \label{pic:2_2_1}
    \end{center}
\end{figure}

\begin{figure}[H]
    \begin{center}
        \includegraphics[width=1\columnwidth]{fig/Lab1/ZeroDiv/2_2_2.png}
        \caption{Информация об обработке исключения в Windbg. GetExceptionCode в блоке \_\_except}
        \label{pic:2_2_2}
    \end{center}
\end{figure}
Получение кода исключения обоими способами было практически идентичным, за исключением обращений к разным регистрам памяти.

\subsubsection{Функция-фильтр}
Представлена функция-фильтр, которая возвращает \textbf{EXCEPTION\_EXECUTE\_HANDLER} в случае, если код зафиксированной ошибки в блоке \textbf{\_\_try}, возвращаемый функцией \textbf{GetExceptionCode}, соотвествует коду ошибки \textbf{EXCEPTION\_INT\_DIVIDE\_BY\_ZERO} иначе функция возвращает
\textbf{EXCEPTION\_CONTINUE\_SEARCH}

\begin{lstlisting}[caption=Реализия функции-фильтра]
#include "stdafx.h"
#include <iostream>
#include <windows.h>
#include <exception>

LONG FilterFunc(DWORD dwExceptionCode) {
    return ((dwExceptionCode == EXCEPTION_INT_DIVIDE_BY_ZERO) ? EXCEPTION_EXECUTE_HANDLER : EXCEPTION_CONTINUE_SEARCH);
}

int main()
{
    __try {
        int x, y = 0;
        x = 5 / y;
    }
    __except (FilterFunc(GetExceptionCode())) {
        std::cout << "Division by zero, own filter function is used";
    }

    std::getchar();
    return 0;
}
\end{lstlisting}

\begin{figure}[H]
    \begin{center}
        \includegraphics[width=1\columnwidth]{fig/Lab1/ZeroDiv/3_1.png}
        \caption{Результат работы программы}
        \label{pic:3_1}
    \end{center}
\end{figure}

\begin{figure}[H]
    \begin{center}
        \includegraphics[width=1\columnwidth]{fig/Lab1/ZeroDiv/3_2_2.png}
        \caption{Информация об обработке исключения в Windbg. Протокол работы программы и стек вызовов}
        \label{pic:3_2_2}
    \end{center}
\end{figure}

При обработке исключения (деление на ноль) происходит вызов специально созданной функции-фильтра \textbf{FilterFunc}, возвращающей в исходную функцию \textbf{main} код продолжения выполнения или код запуска обработчика.

\subsubsection{Ручная генерация исключения и информация об ошибке}
Чтобы получить более подробную информацию об ошибке можно использовать функцию \textbf{GetExceptionInformation}. Данная функция предоставляет следующую информацию об ошибке:
\begin{enumerate}
    \item Код зафиксированной ошибки;
    \item Адрес возникновения исключения;
    \item Виртуальный адрес недоступных данных об ошибке.
\end{enumerate}
Исключение можно возбудить не только в результате каких-то арифметических или логических операций, но и искусственным образом, вызвав функцию \textbf{RaiseException}.
У данной функции есть четыре параметрами, первый из которых определяет тип возбуждаемого исключения.

\begin{lstlisting}[caption=Ручная генерация исключения и информация об ошибке]
#include "stdafx.h"
#include <iostream>
#include <windows.h>
#include <exception>
#include <winerror.h>

int main()
{
    LPEXCEPTION_POINTERS info = nullptr;
    __try {
        int x, y = 0;

        if (y == 0) {
            RaiseException(EXCEPTION_INT_DIVIDE_BY_ZERO, EXCEPTION_NONCONTINUABLE, 0, NULL);
        }

        x = 5 / y;
    }
    __except (info = GetExceptionInformation(), EXCEPTION_EXECUTE_HANDLER) {
        if (info->ExceptionRecord->ExceptionCode == EXCEPTION_INT_DIVIDE_BY_ZERO) {
            std::cout << "Catch manually raised divide by zero exception";
            std::cout << "\nRaised Exception Information: " << info->ExceptionRecord->ExceptionInformation;
            std::cout << "\nRaised Exception Code: " << info->ExceptionRecord->ExceptionCode;
            exit(1);
        }
    }
    return 0;
}
\end{lstlisting}

\begin{figure}[H]
    \begin{center}
        \includegraphics[width=1\columnwidth]{fig/Lab1/ZeroDiv/4_1.png}
        \caption{Результат работы программы}
        \label{pic:4_1}
    \end{center}
\end{figure}

\begin{figure}[H]
    \begin{center}
        \includegraphics[width=1\columnwidth]{fig/Lab1/ZeroDiv/4_2.png}
        \caption{Информация об обработке исключения в Windbg}
        \label{pic:4_2}
    \end{center}
\end{figure}

При генерации программного исключения команда \textbf{RaiseException()} вызывается на уровне ядра.
После этого стандартный обработчик начинает обрабатывать исключение согласно полученному коду.
Информация об исключении, полученная из функции \textbf{GetExceptionInformation} в действительности никакой информацией не владеет, а только возвращает указатель на структуру EXCEPTION\_POINTERS. В свою очередь, эта структура содержит два указателя: \textbf{ExceptionRecord} и \textbf{ContextRecord}.
Важной особенностью функции \textbf{GetExceptionInformation} является то, что ее можно вызывать только в функции-фильтре исключений.
Это обусловлено тем, что структуры CONTEXT, EXCEPTION\_RECORD и EXCEPTION\_POINTERS существуют лишь во время обработки фильтра исключения.

\subsubsection{Необработанное исключение}
Если ни один из установленных программистом обработчиков не подошла для обработки исключения (либо программист вообще не установил ни один обработчик), то вызывается функция \textbf{UnhandledExceptionFilter}, которая выполняет проверку, запущен ли процесс под отладчиком, и информирует процесс, если отладчик доступен. Также стандартную системную функцию-фильтр \textbf{UnhandledExceptionFilter} можно заменить пользовательской функцией-фильтром с помощью функции \textbf{SetUnhandledExceptionFilter}, которая в качестве агрумента принимает пользовательскую функцию-фильтр. Пользовательская функция-фильтр должна иметь прототип, аналогичный функции-фильтру
\begin{lstlisting}[caption=Необработанное исключение]
#include "stdafx.h"
#include <iostream>
#include <stdio.h>
#include <windows.h>
#include <conio.h>
#include <exception>
#include <winerror.h>

int y = 0;

LONG WINAPI UnhandledExceptionFilterFunc(_In_ struct _EXCEPTION_POINTERS* pExceptionPtrs) {
    std::cout << "Caught division by zero\n";
    y++;
    return EXCEPTION_CONTINUE_EXECUTION;
}

int main()
{
    std::cout << "Set filter for unhandled exception\n";
    SetUnhandledExceptionFilter(UnhandledExceptionFilterFunc);

    std::cout << "Filter is set\n";
    int x = 5 / y;
    return 0;
}
\end{lstlisting}

\begin{figure}[H]
    \begin{center}
        \includegraphics[width=1\columnwidth]{fig/Lab1/ZeroDiv/5_1.png}
        \caption{Результат работы программы}
        \label{pic:5_1}
    \end{center}
\end{figure}

\begin{figure}[H]
    \begin{center}
        \includegraphics[width=1\columnwidth]{fig/Lab1/ZeroDiv/5_2.png}
        \caption{Информация об обработке исключения в Windbg}
        \label{pic:5_2}
    \end{center}
\end{figure}

\subsubsection{Вложенные исключения}
Программа реализует вложенный блок \_\_try/\_\_except для обработки исключения деления на ноль.

\begin{lstlisting}[caption=Вложенные исключения]
#include "stdafx.h"
#include <iostream>
#include <windows.h>
#include <exception>
#include <winerror.h>

int main()
{
    DWORD code;
    LPEXCEPTION_POINTERS information;

    __try {
        __try {
            int x, y = 0;
            x = 5 / y;
        }
        __except (code = GetExceptionCode(), information = GetExceptionInformation(), EXCEPTION_EXECUTE_HANDLER) {
            if (code == EXCEPTION_INT_DIVIDE_BY_ZERO) {
                std::cout << "Catch exception divide by zero in inner block \n";
            }
            else {
                EXCEPTION_CONTINUE_SEARCH;
            }
        }
    }
    __except (code = GetExceptionCode(), information = GetExceptionInformation(), EXCEPTION_EXECUTE_HANDLER) {
        if (code == EXCEPTION_INT_DIVIDE_BY_ZERO) {
            std::cout << "Catch exception divide by zero in outer block";
        }
        else {
            EXCEPTION_CONTINUE_SEARCH;
        }
    }

    return 0;
}
\end{lstlisting}

\begin{figure}[H]
    \begin{center}
        \includegraphics[width=1\columnwidth]{fig/Lab1/ZeroDiv/6_1.png}
        \caption{Результат работы программы}
        \label{pic:6_1}
    \end{center}
\end{figure}

\begin{figure}[H]
    \begin{center}
        \includegraphics[width=1\columnwidth]{fig/Lab1/ZeroDiv/6_2.png}
        \caption{Информация об обработке исключения в Windbg}
        \label{pic:6_2}
    \end{center}
\end{figure}

В данном примере исключение было обработано во вложенном блоке \_\_catch, так как ошибка произошла во вложенном \_\_try. Если бы исключение случилось за пределами вложенного блока \_\_try, ошибка была бы обработана на верхнем уровне.
\subsubsection{Оператор goto}
Программа реализует выход из блока \_\_try с помощью оператора \textbf{goto}

\begin{lstlisting}[caption=Выход из блока с помозью оператора goto]
#include "stdafx.h"
#include <windows.h>
#include <exception>
#include <winerror.h>
#include <iostream>

int main()
{
    DWORD code;
    LPEXCEPTION_POINTERS information;

    __try {
        int x = 5, y = 0;
        if (y == 0) {
            std::cout << "Now operator goto will take action\n";
            goto finish;
        } else {
            x = x / y;
        }
    }
    __except (code = GetExceptionCode(), information = GetExceptionInformation(), EXCEPTION_EXECUTE_HANDLER) {
        if (code == EXCEPTION_INT_DIVIDE_BY_ZERO) {
            std::cout << "Caught integer division by zero\n";
            exit(1);
        } else {
            EXCEPTION_CONTINUE_SEARCH;
        }
    }

    finish: std::cout << "Goto completed";
    return 0;
}
\end{lstlisting}

\begin{figure}[H]
    \begin{center}
        \includegraphics[width=1\columnwidth]{fig/Lab1/ZeroDiv/7_1.png}
        \caption{Результат работы программы}
        \label{pic:7_1}
    \end{center}
\end{figure}

\begin{figure}[H]
    \begin{center}
        \includegraphics[width=1\columnwidth]{fig/Lab1/ZeroDiv/7_2_1.png}
        \caption{Информация об обработке исключения в Windbg до раскрутки стека}
        \label{pic:7_2_1}
    \end{center}
\end{figure}

\begin{figure}[H]
    \begin{center}
        \includegraphics[width=1\columnwidth]{fig/Lab1/ZeroDiv/7_2_2.png}
        \caption{Информация об обработке исключения в Windbg после раскрутки стека}
        \label{pic:7_2_2}
    \end{center}
\end{figure}

При вызове оператора \textbf{goto} происходит раскрутка стека, выражающаяся в очистке всех локальных переменных, которые определены в текущем блоке, и безусловный переход к определенной заранее метке. Использование \textbf{goto} может привести к утечкам памяти в процессе раскрутки стека, но в то же время он позволяет сделать переход сразу через несколько участков кода.

\subsubsection{Оператор \_\_leave}
Программа реализует выход из блока \textbf{\_\_try} с помощью оператора \textbf{\_\_leave}

\begin{lstlisting}[caption=Выход из блока с помозью оператора \_\_leave]
#include "stdafx.h"
#include <windows.h>
#include <exception>
#include <winerror.h>
#include <iostream>

int main()
{
    DWORD code;
    LPEXCEPTION_POINTERS information;

    __try {
        int x = 5, y = 0;
        if (y == 0) {
            std::cout << "Now operator goto will take action\n";
            goto finish;
        } else {
            x = x / y;
        }
    }
    __except (code = GetExceptionCode(), information = GetExceptionInformation(), EXCEPTION_EXECUTE_HANDLER) {
        if (code == EXCEPTION_INT_DIVIDE_BY_ZERO) {
            std::cout << "Caught integer division by zero\n";
            exit(1);
        } else {
            EXCEPTION_CONTINUE_SEARCH;
        }
    }

    finish: std::cout << "Goto completed";
    return 0;
}
\end{lstlisting}

\begin{figure}[H]
    \begin{center}
        \includegraphics[width=1\columnwidth]{fig/Lab1/ZeroDiv/8_1.png}
        \caption{Результат работы программы}
        \label{pic:8_1}
    \end{center}
\end{figure}

\begin{figure}[H]
    \begin{center}
        \includegraphics[width=1\columnwidth]{fig/Lab1/ZeroDiv/8_2_2.png}
        \caption{Информация об обработке исключения в Windbg до выхода из блока}
        \label{pic:8_2_2}
    \end{center}
\end{figure}

\begin{figure}[H]
    \begin{center}
        \includegraphics[width=1\columnwidth]{fig/Lab1/ZeroDiv/8_2_3.png}
        \caption{Информация об обработке исключения в Windbg после выхода из блока}
        \label{pic:8_2_3}
    \end{center}
\end{figure}

При вызове оператора \textbf{\_\_leave} происходит завершение работы блока без раскрутки стека, которая выполняется на следующем шаге отладчика. Это главное отличие данного оператора от \textbf{goto}, так как очищение локальных переменных блока снижает производительность программы в целом. После перехода выполняется обработчик завершения.

\subsubsection{Преобразование SEH в C++ исключение}
Программа преобразует структурное исключение в исключение языка С с помощью функции \textbf{translator}
\begin{lstlisting}[caption=Преобразование SEH в исключение С++]
#include "stdafx.h"
#include <iostream>
#include <windows.h>
#include <eh.h>

void run() {
    __try {
        int x, y = 0;
        x = 5 / y;
    }
    __finally {
        std::cout << "In finally block\n";
    }
}

class SE_EXCEPTION {
    private:
    unsigned int nSE;
    public:
    SE_EXCEPTION(unsigned int n) {
        nSE = n;
    }
    unsigned int getSENum() {
        return nSE;
    }
};

void translator(unsigned int n, _EXCEPTION_POINTERS* pEx)
{
    std::cout << "Inside translator function\n";
    throw SE_EXCEPTION(n);
}

int main(void)
{
    try {
        _set_se_translator(translator);
        run();
    } catch (SE_EXCEPTION e) {
        std::cout << "Caught exception after translation";
    }

    return 0;
}
\end{lstlisting}

\begin{figure}[H]
    \begin{center}
        \includegraphics[width=1\columnwidth]{fig/Lab1/ZeroDiv/9_1.png}
        \caption{Результат работы программы}
        \label{pic:9_1}
    \end{center}
\end{figure}

\begin{figure}[H]
    \begin{center}
        \includegraphics[width=1\columnwidth]{fig/Lab1/ZeroDiv/9_2_2.png}
        \caption{Информация об обработке исключения в Windbg внутри функции translator}
        \label{pic:9_2_2}
    \end{center}
\end{figure}
В ходе работы программа был произведен вызов исключения в трансляторе, используя функцию \_set\_se\_translator. После этого происходит выполнение деления на ноль в отдельной функции, с заранее обработанным исключением языка C. Таким образом, получилось структурно обработать исключение с помощью средств языка С.

\subsubsection{Ненормальное выполнение и финальный обработчик}
Финальная обработка исключений используется для того, чтобы при любом исходе исполнения блока \_\_try освободить ресурсы (память, файлы, критические секции и т.п.), которые были захвачены внутри этого блока. Финальный код будет выполняться в любом случае. Во избежание ошибок необходимо проверять завершение блока \_\_try – нормальное или нет.
\begin{lstlisting}[caption=Обработка исключений с использованием блока \_\_finally]
#include "stdafx.h"
#include <iostream>
#include <windows.h>
#include <exception>
#include <winerror.h>

int main()
{
    DWORD code;
    LPEXCEPTION_POINTERS information;

    __try {
        __try {
            int x, y = 0;
            x = 5 / y;
        }
        __finally {
            std::cout << "In finally block\n";
            std::cout << (AbnormalTermination() ? "abnormal\n" : "normal\n");
        }
    }
    __except (code = GetExceptionCode(), information = GetExceptionInformation(), EXCEPTION_EXECUTE_HANDLER) {
        if (code == EXCEPTION_INT_DIVIDE_BY_ZERO) {
            std::cout << "Catch division by zero exception";
            std::getchar();
            exit(1);
        } else {
            EXCEPTION_CONTINUE_SEARCH;
        }
    }

    return 0;
}
\end{lstlisting}

\begin{figure}[H]
    \begin{center}
        \includegraphics[width=1\columnwidth]{fig/Lab1/ZeroDiv/10_1.png}
        \caption{Результат работы программы}
        \label{pic:10_1}
    \end{center}
\end{figure}

\begin{figure}[H]
    \begin{center}
        \includegraphics[width=1\columnwidth]{fig/Lab1/ZeroDiv/10_2_1.png}
        \caption{Информация об обработке исключения в Windbg до блока \_\_finally}
        \label{pic:10_2_1}
    \end{center}
\end{figure}

\begin{figure}[H]
    \begin{center}
        \includegraphics[width=1\columnwidth]{fig/Lab1/ZeroDiv/10_2_2.png}
        \caption{Информация об обработке исключения в Windbg внутри блока \_\_finally}
        \label{pic:10_2_2}
    \end{center}
\end{figure}

\begin{figure}[H]
    \begin{center}
        \includegraphics[width=1\columnwidth]{fig/Lab1/ZeroDiv/10_2_3.png}
        \caption{Информация об обработке исключения в Windbg после блока \_\_finally}
        \label{pic:10_2_3}
    \end{center}
\end{figure}

Проверка на нормальное или ненормальное выполнение программы осуществляется с использованием функции \textbf{AbnormalTermination}. Как видно на Рис. 5.25, блок завершился ненормально из-за исключения деления на ноль.

    \subsection{Переполнение разряда Integer}
Аналогичные задания были выполнены для индивидуального варианта исключения — переполнения разряда Integer. Из-за особенностей \cite{b2, b3} процессора компьютера, на котором выполнялась работа исключение EXCEPTION\_INT\_OVERFLOW не выбрасывалось во время выполнения программы, то есть INT\_MAX становился INT\_MIN без генерации исключения. По этой причине для ряда первых заданий исключение было вызвано искусственно при помощи \textbf{RaiseException}

\subsubsection{Генерация и обработка исключений с помощью функций WinAPI}

\begin{lstlisting}[caption=Генерация и обработка исключения]
#include "stdafx.h"
#include <iostream>
#include <windows.h>
#include <exception>

int main()
{
    __try {
        int x = 0;
        while (true) {
            x++;

            if (x == INT_MIN) {
                RaiseException(EXCEPTION_INT_OVERFLOW, 0, 0, 0);
            }
        }
    }
    __except (EXCEPTION_EXECUTE_HANDLER) {
        std::cout << "Int overflow exception";
    }

    std::getchar();

    return 0;
}
\end{lstlisting}

\begin{figure}[H]
    \begin{center}
        \includegraphics[width=1\columnwidth]{fig/Lab1/IntOverflow/1_1.png}
        \caption{Результат работы программы}
        \label{pic:1_1}
    \end{center}
\end{figure}

\begin{figure}[H]
    \begin{center}
        \includegraphics[width=1\columnwidth]{fig/Lab1/IntOverflow/1_2.png}
        \caption{Информация об обработке исключения в Windbg}
        \label{pic:1_2}
    \end{center}
\end{figure}

\subsubsection{Получение кода исключения}
\begin{lstlisting}[caption=Получение кода исключения с помощью функции GetExceptionCode]
#include "stdafx.h"
#include "stdafx.h"
#include <iostream>
#include <windows.h>
#include <exception>
#include <winerror.h>

int main()
{
    DWORD code = -1;
    __try {
        int x = 0;
        while (true) {
            x++;

            if (x == INT_MIN) {
                RaiseException(EXCEPTION_INT_OVERFLOW, 0, 0, 0);
            }
        }
    }
    __except (code = GetExceptionCode(), ((code == EXCEPTION_INT_OVERFLOW) ? EXCEPTION_EXECUTE_HANDLER : EXCEPTION_CONTINUE_SEARCH)) {
        std::cout << "\nInt overflow inside except, exception code " << code;
    }

    code = -1;

    __try {
        int x = 0;
        while (true) {
            x++;

            if (x == INT_MIN) {
                RaiseException(EXCEPTION_INT_OVERFLOW, 0, 0, 0);
            }
        }
    }
    __except (code = GetExceptionCode(), EXCEPTION_EXECUTE_HANDLER) {
        switch (code) {
            case EXCEPTION_INT_OVERFLOW:
                std::cout << "\nInt overflow inside filter, exception code " << code;
                break;
            default:
                EXCEPTION_CONTINUE_SEARCH;
        }
    }

    std::getchar();

    return 0;
}
\end{lstlisting}

\begin{figure}[H]
    \begin{center}
        \includegraphics[width=1\columnwidth]{fig/Lab1/IntOverflow/2_1.png}
        \caption{Результат работы программы}
        \label{pic:2_1}
    \end{center}
\end{figure}

\begin{figure}[H]
    \begin{center}
        \includegraphics[width=1\columnwidth]{fig/Lab1/IntOverflow/2_2_1.png}
        \caption{Информация об обработке исключения в Windbg. GetExceptionCode в выражении-фильтре}
        \label{pic:2_2_1}
    \end{center}
\end{figure}

\begin{figure}[H]
    \begin{center}
        \includegraphics[width=1\columnwidth]{fig/Lab1/IntOverflow/2_2_2.png}
        \caption{Информация об обработке исключения в Windbg. GetExceptionCode в блоке \_\_except}
        \label{pic:2_2_2}
    \end{center}
\end{figure}

\subsubsection{Функция-фильтр}
\begin{lstlisting}[caption=Реализия функции-фильтра]
#include "stdafx.h"
#include <iostream>
#include <windows.h>
#include <exception>

LONG FilterFunc(DWORD dwExceptionCode) {
    return ((dwExceptionCode == EXCEPTION_INT_OVERFLOW) ? EXCEPTION_EXECUTE_HANDLER : EXCEPTION_CONTINUE_SEARCH);
}

int main()
{
    __try {
        int x = 0;
        while (true) {
            x++;

            if (x == INT_MIN) {
                RaiseException(EXCEPTION_INT_OVERFLOW, 0, 0, 0);
            }
        }
    }
    __except (FilterFunc(GetExceptionCode())) {
        std::cout << "Int overflow, own filter function is used";
    }

    std::getchar();
    return 0;
}
\end{lstlisting}

\begin{figure}[H]
    \begin{center}
        \includegraphics[width=1\columnwidth]{fig/Lab1/IntOverflow/3_1.png}
        \caption{Результат работы программы}
        \label{pic:3_1}
    \end{center}
\end{figure}

\begin{figure}[H]
    \begin{center}
        \includegraphics[width=1\columnwidth]{fig/Lab1/IntOverflow/3_2.png}
        \caption{Информация об обработке исключения в Windbg. Протокол работы программы и стек вызовов}
        \label{pic:3_2_2}
    \end{center}
\end{figure}

\subsubsection{Ручная генерация исключения и информация об ошибке}

\begin{lstlisting}[caption=Ручная генерация исключения и информация об ошибке]
#include "stdafx.h"
#include <iostream>
#include <windows.h>
#include <exception>
#include <winerror.h>

int main()
{
    LPEXCEPTION_POINTERS info = nullptr;
    __try {
        int x = 0;
        while (true) {
            x++;

            if (x == INT_MIN) {
                RaiseException(EXCEPTION_INT_OVERFLOW, 0, 0, 0);
            }
        }
    }
    __except (info = GetExceptionInformation(), EXCEPTION_EXECUTE_HANDLER) {
        if (info->ExceptionRecord->ExceptionCode == EXCEPTION_INT_OVERFLOW) {
            std::cout << "Catch manually raised int overflow exception";
            std::cout << "\nRaised Exception Information: " << info->ExceptionRecord->ExceptionInformation;
            std::cout << "\nRaised Exception Code: " << info->ExceptionRecord->ExceptionCode;
            std::getchar();
            exit(1);
        }
    }
    return 0;
}
\end{lstlisting}

\begin{figure}[H]
    \begin{center}
        \includegraphics[width=1\columnwidth]{fig/Lab1/IntOverflow/4_1.png}
        \caption{Результат работы программы}
        \label{pic:4_1}
    \end{center}
\end{figure}

\begin{figure}[H]
    \begin{center}
        \includegraphics[width=1\columnwidth]{fig/Lab1/IntOverflow/4_2.png}
        \caption{Информация об обработке исключения в Windbg}
        \label{pic:4_2}
    \end{center}
\end{figure}

\subsubsection{Необработанное исключение}
\begin{lstlisting}[caption=Необработанное исключение]
#include "stdafx.h"
#include "stdafx.h"
#include <iostream>
#include <stdio.h>
#include <windows.h>
#include <conio.h>
#include <exception>
#include <winerror.h>

int x = 0;

LONG WINAPI UnhandledExceptionFilterFunc(_In_ struct _EXCEPTION_POINTERS* pExceptionPtrs) {
    std::cout << "Caught int overflow\n";
    x++;
    return EXCEPTION_CONTINUE_EXECUTION;
}

int main()
{
    std::cout << "Set filter for unhandled exception\n";
    SetUnhandledExceptionFilter(UnhandledExceptionFilterFunc);

    std::cout << "Filter is set\n";
    while (true) {
        x++;

        if (x == INT_MIN) {
            RaiseException(EXCEPTION_INT_OVERFLOW, 0, 0, 0);
        }
    }
    return 0;
}
\end{lstlisting}

\begin{figure}[H]
    \begin{center}
        \includegraphics[width=1\columnwidth]{fig/Lab1/IntOverflow/5_1.png}
        \caption{Результат работы программы}
        \label{pic:5_1}
    \end{center}
\end{figure}

\begin{figure}[H]
    \begin{center}
        \includegraphics[width=1\columnwidth]{fig/Lab1/IntOverflow/5_2.png}
        \caption{Информация об обработке исключения в Windbg}
        \label{pic:5_2}
    \end{center}
\end{figure}

\subsubsection{Вложенные исключения}
Программа реализует вложенный блок \_\_try/\_\_except для обработки исключения деления на ноль.

\begin{lstlisting}[caption=Вложенные исключения]
#include "stdafx.h"
#include <iostream>
#include <windows.h>
#include <exception>
#include <winerror.h>

int main()
{
    DWORD code;
    LPEXCEPTION_POINTERS information;

    __try {
        __try {
            int x = 0;
            while (true) {
                x++;

                if (x == INT_MIN) {
                    RaiseException(EXCEPTION_INT_OVERFLOW, 0, 0, 0);
                }
            }
        }
        __except (code = GetExceptionCode(), information = GetExceptionInformation(), EXCEPTION_EXECUTE_HANDLER) {
            if (code == EXCEPTION_INT_OVERFLOW) {
                std::cout << "Catch int overflow exception in inner block \n";
            }
            else {
                EXCEPTION_CONTINUE_SEARCH;
            }
        }
    }
    __except (code = GetExceptionCode(), information = GetExceptionInformation(), EXCEPTION_EXECUTE_HANDLER) {
        if (code == EXCEPTION_INT_OVERFLOW) {
            std::cout << "Catch int overflow exception in outer block";
        }
        else {
            EXCEPTION_CONTINUE_SEARCH;
        }
    }

    std::getchar();
    return 0;
}
\end{lstlisting}

\begin{figure}[H]
    \begin{center}
        \includegraphics[width=1\columnwidth]{fig/Lab1/IntOverflow/6_1.png}
        \caption{Результат работы программы}
        \label{pic:6_1}
    \end{center}
\end{figure}

\begin{figure}[H]
    \begin{center}
        \includegraphics[width=1\columnwidth]{fig/Lab1/IntOverflow/6_2.png}
        \caption{Информация об обработке исключения в Windbg}
        \label{pic:6_2}
    \end{center}
\end{figure}

\subsubsection{Оператор goto}
\begin{lstlisting}[caption=Выход из блока с помозью оператора goto]
#include "stdafx.h"
#include <windows.h>
#include <exception>
#include <winerror.h>
#include <iostream>

int main()
{
    DWORD code;
    LPEXCEPTION_POINTERS information;

    __try {
        int x = INT_MAX;
        if (++x == INT_MIN) {
            std::cout << "Now operator goto will take action\n";
            goto finish;
        }
        else {
            x++;
        }
    }
    __except (code = GetExceptionCode(), information = GetExceptionInformation(), EXCEPTION_EXECUTE_HANDLER) {
        if (code == EXCEPTION_INT_OVERFLOW) {
            std::cout << "Caught int overflow exception\n";
            exit(1);
        }
        else {
            EXCEPTION_CONTINUE_SEARCH;
        }
    }

    finish: std::cout << "Goto completed";
    std::getchar();
    return 0;
}
\end{lstlisting}

\begin{figure}[H]
    \begin{center}
        \includegraphics[width=1\columnwidth]{fig/Lab1/IntOverflow/7_1.png}
        \caption{Результат работы программы}
        \label{pic:7_1}
    \end{center}
\end{figure}

\begin{figure}[H]
    \begin{center}
        \includegraphics[width=1\columnwidth]{fig/Lab1/IntOverflow/7_2_1.png}
        \caption{Информация об обработке исключения в Windbg до раскрутки стека}
        \label{pic:7_2_1}
    \end{center}
\end{figure}

\begin{figure}[H]
    \begin{center}
        \includegraphics[width=1\columnwidth]{fig/Lab1/IntOverflow/7_2_2.png}
        \caption{Информация об обработке исключения в Windbg после раскрутки стека}
        \label{pic:7_2_2}
    \end{center}
\end{figure}

\subsubsection{Оператор \_\_leave}
\begin{lstlisting}[caption=Выход из блока с помозью оператора \_\_leave]
#include "stdafx.h"
#include <iostream>
#include <windows.h>
#include <exception>
#include <winerror.h>

int main()
{
    DWORD code;
    LPEXCEPTION_POINTERS information;

    __try {
        int x = INT_MAX;
        if (++x == INT_MIN) {
            std::cout << "Now leave operation will take action";
            __leave;
        }
        else {
            x++;
        }
    }
    __except (code = GetExceptionCode(), information = GetExceptionInformation(), EXCEPTION_EXECUTE_HANDLER) {
        if (code == EXCEPTION_INT_OVERFLOW) {
            std::cout << "Catch int overflow exception";
            exit(1);
        }
        else {
            EXCEPTION_CONTINUE_SEARCH;
        }
    }

    std::getchar();
    return 0;
}
\end{lstlisting}

\begin{figure}[H]
    \begin{center}
        \includegraphics[width=1\columnwidth]{fig/Lab1/IntOverflow/8_1.png}
        \caption{Результат работы программы}
        \label{pic:8_1}
    \end{center}
\end{figure}

\begin{figure}[H]
    \begin{center}
        \includegraphics[width=1\columnwidth]{fig/Lab1/IntOverflow/8_2_1.png}
        \caption{Информация об обработке исключения в Windbg до выхода из блока}
        \label{pic:8_2_2}
    \end{center}
\end{figure}

\begin{figure}[H]
    \begin{center}
        \includegraphics[width=1\columnwidth]{fig/Lab1/IntOverflow/8_2_2.png}
        \caption{Информация об обработке исключения в Windbg после выхода из блока}
        \label{pic:8_2_3}
    \end{center}
\end{figure}

\subsubsection{Преобразование SEH в C++ исключение}
\begin{lstlisting}[caption=Преобразование SEH в исключение С++]
#include "stdafx.h"
#include <iostream>
#include <windows.h>
#include <eh.h>

void run() {
    __try {
        int x = 0;
        while (true) {
            x++;

            if (x == INT_MIN) {
                RaiseException(EXCEPTION_INT_OVERFLOW, 0, 0, 0);
            }
        }
    }
    __finally {
        std::cout << "In finally block\n";
    }
}

class SE_EXCEPTION {
    private:
    unsigned int nSE;
    public:
    SE_EXCEPTION(unsigned int n) {
        nSE = n;
    }
    unsigned int getSENum() {
        return nSE;
    }
};

void translator(unsigned int n, _EXCEPTION_POINTERS* pEx)
{
    std::cout << "Inside translator function\n";

    throw SE_EXCEPTION(n);
}

int main(void)
{
    try {
        _set_se_translator(translator);
        run();
    }
    catch (SE_EXCEPTION e) {
        std::cout << "Caught exception after translation";
    }

    std::getchar();

    return 0;
}
\end{lstlisting}

\begin{figure}[H]
    \begin{center}
        \includegraphics[width=1\columnwidth]{fig/Lab1/IntOverflow/9_1.png}
        \caption{Результат работы программы}
        \label{pic:9_1}
    \end{center}
\end{figure}

\begin{figure}[H]
    \begin{center}
        \includegraphics[width=1\columnwidth]{fig/Lab1/IntOverflow/9_2_2.png}
        \caption{Информация об обработке исключения в Windbg внутри функции translator}
        \label{pic:9_2_2}
    \end{center}
\end{figure}

\subsubsection{Ненормальное выполнение и финальный обработчик}
\begin{lstlisting}[caption=Обработка исключений с использованием блока \_\_finally]
#include "stdafx.h"
#include <iostream>
#include <windows.h>
#include <exception>
#include <winerror.h>

int main()
{
    DWORD code;
    LPEXCEPTION_POINTERS information;

    __try {
        __try {
            int x = 0;
            while (true) {
                x++;

                if (x == INT_MIN) {
                    RaiseException(EXCEPTION_INT_OVERFLOW, 0, 0, 0);
                }
            }
        }
        __finally {
            std::cout << "In finally block\n";
            std::cout << (AbnormalTermination() ? "abnormal\n" : "normal\n");
        }
    }
    __except (code = GetExceptionCode(), information = GetExceptionInformation(), EXCEPTION_EXECUTE_HANDLER) {
        if (code == EXCEPTION_INT_OVERFLOW) {
            std::cout << "Catch int overflow exception";
            std::getchar();
            exit(1);
        }
        else {
            EXCEPTION_CONTINUE_SEARCH;
        }
    }

    return 0;
}
\end{lstlisting}

\begin{figure}[H]
    \begin{center}
        \includegraphics[width=1\columnwidth]{fig/Lab1/IntOverflow/10_1.png}
        \caption{Результат работы программы}
        \label{pic:10_1}
    \end{center}
\end{figure}

\begin{figure}[H]
    \begin{center}
        \includegraphics[width=1\columnwidth]{fig/Lab1/IntOverflow/10_2_1.png}
        \caption{Информация об обработке исключения в Windbg до блока \_\_finally}
        \label{pic:10_2_1}
    \end{center}
\end{figure}

\begin{figure}[H]
    \begin{center}
        \includegraphics[width=1\columnwidth]{fig/Lab1/IntOverflow/10_2_2.png}
        \caption{Информация об обработке исключения в Windbg внутри блока \_\_finally}
        \label{pic:10_2_2}
    \end{center}
\end{figure}

\begin{figure}[H]
    \begin{center}
        \includegraphics[width=1\columnwidth]{fig/Lab1/IntOverflow/10_2_3.png}
        \caption{Информация об обработке исключения в Windbg после блока \_\_finally}
        \label{pic:10_2_3}
    \end{center}
\end{figure}

    \section{Выводы}
    В ходе данной работы были изучены принципы структурной обработки исключений (SEH) с использованием специальных методов и средств, предоставляемых библиотекой WinAPI. При помощи средства отладки WinDbg были исследованы системные процессы, происходящие при генерации и обработке исключений. Несмотря на то, что SEH и исключения в языке C++, на первый взгляд, могут показаться идентичными, они сильно отличаются друг от друга.
    Их совместное применение требует внимательного обращения, поскольку обработчики исключений, написанные пользователем и сгенерированные C++, могут взаимодействовать между собой и приводить к нежелательным последствиям. Например, обработчики исключений Windows не осуществляют вызов деструкторов, что в ряде случаев необходимо для уничтожения объектов С++. Документация Microsoft рекомендует полностью отказаться от использования обработчиков Windows в прикладных программах на С++ и ограничиться применением в них только обработчиков исключений С++

    \begin{thebibliography}{00}
        \bibitem{b1} WinDbg, URL https://en.wikipedia.org/wiki/WinDbg;
        \bibitem{b2} Wikipedia, Interger Overflow Flags, URL: https://en.wikipedia.org/wiki/Integer\_overflow
        \bibitem{b3} Intel Developer Zone: How best to handle integer overflow situations, URL: https://software.intel.com/en-us/forums/intel-fortran-compiler/topic/731539
    \end{thebibliography}
\end{document}
