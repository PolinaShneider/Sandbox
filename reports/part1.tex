\subsection{Обработка деления на ноль}
\subsubsection{Генерация и обработка исключений с помощью функций WinAPI}

\begin{lstlisting}[caption=Генерация и обработка исключения]
#include "stdafx.h"
#include <iostream>
#include <windows.h>
#include <exception>

int main()
{
    __try {
        int x, y = 0;
        x = 5 / y;
    } __except (EXCEPTION_EXECUTE_FAULT) {
        std::cout << "Division by zero";
    }

    return 0;
}
\end{lstlisting}

\begin{figure}[H]
    \begin{center}
        \includegraphics[width=1\columnwidth]{fig/Lab1/ZeroDiv/1_1.png}
        \caption{Результат работы программы}
        \label{pic:1_1}
    \end{center}
\end{figure}

\begin{figure}[H]
    \begin{center}
        \includegraphics[width=1\columnwidth]{fig/Lab1/ZeroDiv/1_2.png}
        \caption{Информация об обработке исключения в Windbg}
        \label{pic:1_2}
    \end{center}
\end{figure}

Из рисунка 5.2 видно, что исключение деления на ноль успешно обработано (first chance).

\subsubsection{Получение кода исключения}
Программа реализует отображение кода исключения двумя способами: в выражении фильтра и в блоке \_\_except обработчика прерывания.
За получение кода исключения отвечает функция \textbf{GetExceptionCode()}

В листинге №3 показана работа с данной функцией, которая используется в выражении-фильтре блока \textbf{\_\_except} и непосредственно в самом блоке \textbf{\_\_except}.
\begin{lstlisting}[caption=Получение кода исключения с помощью функции GetExceptionCode]
#include "stdafx.h"
#include <iostream>
#include <windows.h>
#include <exception>
#include <winerror.h>

int main()
{
    DWORD code = -1;
    __try {
        int x, y = 0;
        x = 5 / y;
    }
    __except (code = GetExceptionCode(), ((code == EXCEPTION_INT_DIVIDE_BY_ZERO) ? EXCEPTION_EXECUTE_HANDLER : EXCEPTION_CONTINUE_SEARCH)) {
        std::cout << "Division by zero inside except, exception code " << code;
    }

    code = -1;

    __try {
        int x, y = 0;
        x = 5 / y;
    }
    __except (code = GetExceptionCode(), EXCEPTION_EXECUTE_HANDLER) {
        switch (code) {
            case EXCEPTION_INT_DIVIDE_BY_ZERO:
                std::cout << "\nDivision by zero inside filter, exception code " << code;
                break;
            default:
                EXCEPTION_CONTINUE_SEARCH;
        }
    }

    std::getchar();

    return 0;
}
\end{lstlisting}

\begin{figure}[H]
    \begin{center}
        \includegraphics[width=1\columnwidth]{fig/Lab1/ZeroDiv/2_1.png}
        \caption{Результат работы программы}
        \label{pic:2_1}
    \end{center}
\end{figure}

\begin{figure}[H]
    \begin{center}
        \includegraphics[width=1\columnwidth]{fig/Lab1/ZeroDiv/2_2_1.png}
        \caption{Информация об обработке исключения в Windbg. GetExceptionCode в выражении-фильтре}
        \label{pic:2_2_1}
    \end{center}
\end{figure}

\begin{figure}[H]
    \begin{center}
        \includegraphics[width=1\columnwidth]{fig/Lab1/ZeroDiv/2_2_2.png}
        \caption{Информация об обработке исключения в Windbg. GetExceptionCode в блоке \_\_except}
        \label{pic:2_2_2}
    \end{center}
\end{figure}
Получение кода исключения обоими способами было практически идентичным, за исключением обращений к разным регистрам памяти.

\subsubsection{Функция-фильтр}
Представлена функция-фильтр, которая возвращает \textbf{EXCEPTION\_EXECUTE\_HANDLER} в случае, если код зафиксированной ошибки в блоке \textbf{\_\_try}, возвращаемый функцией \textbf{GetExceptionCode}, соотвествует коду ошибки \textbf{EXCEPTION\_INT\_DIVIDE\_BY\_ZERO} иначе функция возвращает
\textbf{EXCEPTION\_CONTINUE\_SEARCH}

\begin{lstlisting}[caption=Реализия функции-фильтра]
#include "stdafx.h"
#include <iostream>
#include <windows.h>
#include <exception>

LONG FilterFunc(DWORD dwExceptionCode) {
    return ((dwExceptionCode == EXCEPTION_INT_DIVIDE_BY_ZERO) ? EXCEPTION_EXECUTE_HANDLER : EXCEPTION_CONTINUE_SEARCH);
}

int main()
{
    __try {
        int x, y = 0;
        x = 5 / y;
    }
    __except (FilterFunc(GetExceptionCode())) {
        std::cout << "Division by zero, own filter function is used";
    }

    std::getchar();
    return 0;
}
\end{lstlisting}

\begin{figure}[H]
    \begin{center}
        \includegraphics[width=1\columnwidth]{fig/Lab1/ZeroDiv/3_1.png}
        \caption{Результат работы программы}
        \label{pic:3_1}
    \end{center}
\end{figure}

\begin{figure}[H]
    \begin{center}
        \includegraphics[width=1\columnwidth]{fig/Lab1/ZeroDiv/3_2_2.png}
        \caption{Информация об обработке исключения в Windbg. Протокол работы программы и стек вызовов}
        \label{pic:3_2_2}
    \end{center}
\end{figure}

При обработке исключения (деление на ноль) происходит вызов специально созданной функции-фильтра \textbf{FilterFunc}, возвращающей в исходную функцию \textbf{main} код продолжения выполнения или код запуска обработчика.

\subsubsection{Ручная генерация исключения и информация об ошибке}
Чтобы получить более подробную информацию об ошибке можно использовать функцию \textbf{GetExceptionInformation}. Данная функция предоставляет следующую информацию об ошибке:
\begin{enumerate}
    \item Код зафиксированной ошибки;
    \item Адрес возникновения исключения;
    \item Виртуальный адрес недоступных данных об ошибке.
\end{enumerate}
Исключение можно возбудить не только в результате каких-то арифметических или логических операций, но и искусственным образом, вызвав функцию \textbf{RaiseException}.
У данной функции есть четыре параметрами, первый из которых определяет тип возбуждаемого исключения.

\begin{lstlisting}[caption=Ручная генерация исключения и информация об ошибке]
#include "stdafx.h"
#include <iostream>
#include <windows.h>
#include <exception>
#include <winerror.h>

int main()
{
    LPEXCEPTION_POINTERS info = nullptr;
    __try {
        int x, y = 0;

        if (y == 0) {
            RaiseException(EXCEPTION_INT_DIVIDE_BY_ZERO, EXCEPTION_NONCONTINUABLE, 0, NULL);
        }

        x = 5 / y;
    }
    __except (info = GetExceptionInformation(), EXCEPTION_EXECUTE_HANDLER) {
        if (info->ExceptionRecord->ExceptionCode == EXCEPTION_INT_DIVIDE_BY_ZERO) {
            std::cout << "Catch manually raised divide by zero exception";
            std::cout << "\nRaised Exception Information: " << info->ExceptionRecord->ExceptionInformation;
            std::cout << "\nRaised Exception Code: " << info->ExceptionRecord->ExceptionCode;
            exit(1);
        }
    }
    return 0;
}
\end{lstlisting}

\begin{figure}[H]
    \begin{center}
        \includegraphics[width=1\columnwidth]{fig/Lab1/ZeroDiv/4_1.png}
        \caption{Результат работы программы}
        \label{pic:4_1}
    \end{center}
\end{figure}

\begin{figure}[H]
    \begin{center}
        \includegraphics[width=1\columnwidth]{fig/Lab1/ZeroDiv/4_2.png}
        \caption{Информация об обработке исключения в Windbg}
        \label{pic:4_2}
    \end{center}
\end{figure}

При генерации программного исключения команда \textbf{RaiseException()} вызывается на уровне ядра.
После этого стандартный обработчик начинает обрабатывать исключение согласно полученному коду.
Информация об исключении, полученная из функции \textbf{GetExceptionInformation} в действительности никакой информацией не владеет, а только возвращает указатель на структуру EXCEPTION\_POINTERS. В свою очередь, эта структура содержит два указателя: \textbf{ExceptionRecord} и \textbf{ContextRecord}.
Важной особенностью функции \textbf{GetExceptionInformation} является то, что ее можно вызывать только в функции-фильтре исключений.
Это обусловлено тем, что структуры CONTEXT, EXCEPTION\_RECORD и EXCEPTION\_POINTERS существуют лишь во время обработки фильтра исключения.

\subsubsection{Необработанное исключение}
Если ни один из установленных программистом обработчиков не подошла для обработки исключения (либо программист вообще не установил ни один обработчик), то вызывается функция \textbf{UnhandledExceptionFilter}, которая выполняет проверку, запущен ли процесс под отладчиком, и информирует процесс, если отладчик доступен. Также стандартную системную функцию-фильтр \textbf{UnhandledExceptionFilter} можно заменить пользовательской функцией-фильтром с помощью функции \textbf{SetUnhandledExceptionFilter}, которая в качестве агрумента принимает пользовательскую функцию-фильтр. Пользовательская функция-фильтр должна иметь прототип, аналогичный функции-фильтру
\begin{lstlisting}[caption=Необработанное исключение]
#include "stdafx.h"
#include <iostream>
#include <stdio.h>
#include <windows.h>
#include <conio.h>
#include <exception>
#include <winerror.h>

int y = 0;

LONG WINAPI UnhandledExceptionFilterFunc(_In_ struct _EXCEPTION_POINTERS* pExceptionPtrs) {
    std::cout << "Caught division by zero\n";
    y++;
    return EXCEPTION_CONTINUE_EXECUTION;
}

int main()
{
    std::cout << "Set filter for unhandled exception\n";
    SetUnhandledExceptionFilter(UnhandledExceptionFilterFunc);

    std::cout << "Filter is set\n";
    int x = 5 / y;
    return 0;
}
\end{lstlisting}

\begin{figure}[H]
    \begin{center}
        \includegraphics[width=1\columnwidth]{fig/Lab1/ZeroDiv/5_1.png}
        \caption{Результат работы программы}
        \label{pic:5_1}
    \end{center}
\end{figure}

\begin{figure}[H]
    \begin{center}
        \includegraphics[width=1\columnwidth]{fig/Lab1/ZeroDiv/5_2.png}
        \caption{Информация об обработке исключения в Windbg}
        \label{pic:5_2}
    \end{center}
\end{figure}

\subsubsection{Вложенные исключения}
Программа реализует вложенный блок \_\_try/\_\_except для обработки исключения деления на ноль.

\begin{lstlisting}[caption=Вложенные исключения]
#include "stdafx.h"
#include <iostream>
#include <windows.h>
#include <exception>
#include <winerror.h>

int main()
{
    DWORD code;
    LPEXCEPTION_POINTERS information;

    __try {
        __try {
            int x, y = 0;
            x = 5 / y;
        }
        __except (code = GetExceptionCode(), information = GetExceptionInformation(), EXCEPTION_EXECUTE_HANDLER) {
            if (code == EXCEPTION_INT_DIVIDE_BY_ZERO) {
                std::cout << "Catch exception divide by zero in inner block \n";
            }
            else {
                EXCEPTION_CONTINUE_SEARCH;
            }
        }
    }
    __except (code = GetExceptionCode(), information = GetExceptionInformation(), EXCEPTION_EXECUTE_HANDLER) {
        if (code == EXCEPTION_INT_DIVIDE_BY_ZERO) {
            std::cout << "Catch exception divide by zero in outer block";
        }
        else {
            EXCEPTION_CONTINUE_SEARCH;
        }
    }

    return 0;
}
\end{lstlisting}

\begin{figure}[H]
    \begin{center}
        \includegraphics[width=1\columnwidth]{fig/Lab1/ZeroDiv/6_1.png}
        \caption{Результат работы программы}
        \label{pic:6_1}
    \end{center}
\end{figure}

\begin{figure}[H]
    \begin{center}
        \includegraphics[width=1\columnwidth]{fig/Lab1/ZeroDiv/6_2.png}
        \caption{Информация об обработке исключения в Windbg}
        \label{pic:6_2}
    \end{center}
\end{figure}

В данном примере исключение было обработано во вложенном блоке \_\_catch, так как ошибка произошла во вложенном \_\_try. Если бы исключение случилось за пределами вложенного блока \_\_try, ошибка была бы обработана на верхнем уровне.
\subsubsection{Оператор goto}
Программа реализует выход из блока \_\_try с помощью оператора \textbf{goto}

\begin{lstlisting}[caption=Выход из блока с помозью оператора goto]
#include "stdafx.h"
#include <windows.h>
#include <exception>
#include <winerror.h>
#include <iostream>

int main()
{
    DWORD code;
    LPEXCEPTION_POINTERS information;

    __try {
        int x = 5, y = 0;
        if (y == 0) {
            std::cout << "Now operator goto will take action\n";
            goto finish;
        } else {
            x = x / y;
        }
    }
    __except (code = GetExceptionCode(), information = GetExceptionInformation(), EXCEPTION_EXECUTE_HANDLER) {
        if (code == EXCEPTION_INT_DIVIDE_BY_ZERO) {
            std::cout << "Caught integer division by zero\n";
            exit(1);
        } else {
            EXCEPTION_CONTINUE_SEARCH;
        }
    }

    finish: std::cout << "Goto completed";
    return 0;
}
\end{lstlisting}

\begin{figure}[H]
    \begin{center}
        \includegraphics[width=1\columnwidth]{fig/Lab1/ZeroDiv/7_1.png}
        \caption{Результат работы программы}
        \label{pic:7_1}
    \end{center}
\end{figure}

\begin{figure}[H]
    \begin{center}
        \includegraphics[width=1\columnwidth]{fig/Lab1/ZeroDiv/7_2_1.png}
        \caption{Информация об обработке исключения в Windbg до раскрутки стека}
        \label{pic:7_2_1}
    \end{center}
\end{figure}

\begin{figure}[H]
    \begin{center}
        \includegraphics[width=1\columnwidth]{fig/Lab1/ZeroDiv/7_2_2.png}
        \caption{Информация об обработке исключения в Windbg после раскрутки стека}
        \label{pic:7_2_2}
    \end{center}
\end{figure}

При вызове оператора \textbf{goto} происходит раскрутка стека, выражающаяся в очистке всех локальных переменных, которые определены в текущем блоке, и безусловный переход к определенной заранее метке. Использование \textbf{goto} может привести к утечкам памяти в процессе раскрутки стека, но в то же время он позволяет сделать переход сразу через несколько участков кода.

\subsubsection{Оператор \_\_leave}
Программа реализует выход из блока \textbf{\_\_try} с помощью оператора \textbf{\_\_leave}

\begin{lstlisting}[caption=Выход из блока с помозью оператора \_\_leave]
#include "stdafx.h"
#include <windows.h>
#include <exception>
#include <winerror.h>
#include <iostream>

int main()
{
    DWORD code;
    LPEXCEPTION_POINTERS information;

    __try {
        int x = 5, y = 0;
        if (y == 0) {
            std::cout << "Now operator goto will take action\n";
            goto finish;
        } else {
            x = x / y;
        }
    }
    __except (code = GetExceptionCode(), information = GetExceptionInformation(), EXCEPTION_EXECUTE_HANDLER) {
        if (code == EXCEPTION_INT_DIVIDE_BY_ZERO) {
            std::cout << "Caught integer division by zero\n";
            exit(1);
        } else {
            EXCEPTION_CONTINUE_SEARCH;
        }
    }

    finish: std::cout << "Goto completed";
    return 0;
}
\end{lstlisting}

\begin{figure}[H]
    \begin{center}
        \includegraphics[width=1\columnwidth]{fig/Lab1/ZeroDiv/8_1.png}
        \caption{Результат работы программы}
        \label{pic:8_1}
    \end{center}
\end{figure}

\begin{figure}[H]
    \begin{center}
        \includegraphics[width=1\columnwidth]{fig/Lab1/ZeroDiv/8_2_2.png}
        \caption{Информация об обработке исключения в Windbg до выхода из блока}
        \label{pic:8_2_2}
    \end{center}
\end{figure}

\begin{figure}[H]
    \begin{center}
        \includegraphics[width=1\columnwidth]{fig/Lab1/ZeroDiv/8_2_3.png}
        \caption{Информация об обработке исключения в Windbg после выхода из блока}
        \label{pic:8_2_3}
    \end{center}
\end{figure}

При вызове оператора \textbf{\_\_leave} происходит завершение работы блока без раскрутки стека, которая выполняется на следующем шаге отладчика. Это главное отличие данного оператора от \textbf{goto}, так как очищение локальных переменных блока снижает производительность программы в целом. После перехода выполняется обработчик завершения.

\subsubsection{Преобразование SEH в C++ исключение}
Программа преобразует структурное исключение в исключение языка С с помощью функции \textbf{translator}
\begin{lstlisting}[caption=Преобразование SEH в исключение С++]
#include "stdafx.h"
#include <iostream>
#include <windows.h>
#include <eh.h>

void run() {
    __try {
        int x, y = 0;
        x = 5 / y;
    }
    __finally {
        std::cout << "In finally block\n";
    }
}

class SE_EXCEPTION {
    private:
    unsigned int nSE;
    public:
    SE_EXCEPTION(unsigned int n) {
        nSE = n;
    }
    unsigned int getSENum() {
        return nSE;
    }
};

void translator(unsigned int n, _EXCEPTION_POINTERS* pEx)
{
    std::cout << "Inside translator function\n";
    throw SE_EXCEPTION(n);
}

int main(void)
{
    try {
        _set_se_translator(translator);
        run();
    } catch (SE_EXCEPTION e) {
        std::cout << "Caught exception after translation";
    }

    return 0;
}
\end{lstlisting}

\begin{figure}[H]
    \begin{center}
        \includegraphics[width=1\columnwidth]{fig/Lab1/ZeroDiv/9_1.png}
        \caption{Результат работы программы}
        \label{pic:9_1}
    \end{center}
\end{figure}

\begin{figure}[H]
    \begin{center}
        \includegraphics[width=1\columnwidth]{fig/Lab1/ZeroDiv/9_2_2.png}
        \caption{Информация об обработке исключения в Windbg внутри функции translator}
        \label{pic:9_2_2}
    \end{center}
\end{figure}
В ходе работы программа был произведен вызов исключения в трансляторе, используя функцию \_set\_se\_translator. После этого происходит выполнение деления на ноль в отдельной функции, с заранее обработанным исключением языка C. Таким образом, получилось структурно обработать исключение с помощью средств языка С.

\subsubsection{Ненормальное выполнение и финальный обработчик}
Финальная обработка исключений используется для того, чтобы при любом исходе исполнения блока \_\_try освободить ресурсы (память, файлы, критические секции и т.п.), которые были захвачены внутри этого блока. Финальный код будет выполняться в любом случае. Во избежание ошибок необходимо проверять завершение блока \_\_try – нормальное или нет.
\begin{lstlisting}[caption=Обработка исключений с использованием блока \_\_finally]
#include "stdafx.h"
#include <iostream>
#include <windows.h>
#include <exception>
#include <winerror.h>

int main()
{
    DWORD code;
    LPEXCEPTION_POINTERS information;

    __try {
        __try {
            int x, y = 0;
            x = 5 / y;
        }
        __finally {
            std::cout << "In finally block\n";
            std::cout << (AbnormalTermination() ? "abnormal\n" : "normal\n");
        }
    }
    __except (code = GetExceptionCode(), information = GetExceptionInformation(), EXCEPTION_EXECUTE_HANDLER) {
        if (code == EXCEPTION_INT_DIVIDE_BY_ZERO) {
            std::cout << "Catch division by zero exception";
            std::getchar();
            exit(1);
        } else {
            EXCEPTION_CONTINUE_SEARCH;
        }
    }

    return 0;
}
\end{lstlisting}

\begin{figure}[H]
    \begin{center}
        \includegraphics[width=1\columnwidth]{fig/Lab1/ZeroDiv/10_1.png}
        \caption{Результат работы программы}
        \label{pic:10_1}
    \end{center}
\end{figure}

\begin{figure}[H]
    \begin{center}
        \includegraphics[width=1\columnwidth]{fig/Lab1/ZeroDiv/10_2_1.png}
        \caption{Информация об обработке исключения в Windbg до блока \_\_finally}
        \label{pic:10_2_1}
    \end{center}
\end{figure}

\begin{figure}[H]
    \begin{center}
        \includegraphics[width=1\columnwidth]{fig/Lab1/ZeroDiv/10_2_2.png}
        \caption{Информация об обработке исключения в Windbg внутри блока \_\_finally}
        \label{pic:10_2_2}
    \end{center}
\end{figure}

\begin{figure}[H]
    \begin{center}
        \includegraphics[width=1\columnwidth]{fig/Lab1/ZeroDiv/10_2_3.png}
        \caption{Информация об обработке исключения в Windbg после блока \_\_finally}
        \label{pic:10_2_3}
    \end{center}
\end{figure}

Проверка на нормальное или ненормальное выполнение программы осуществляется с использованием функции \textbf{AbnormalTermination}. Как видно на Рис. 5.25, блок завершился ненормально из-за исключения деления на ноль.
